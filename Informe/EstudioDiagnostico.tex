\chapter{Estudio de Diagnóstico}

2.- Esta propuesta nace del Sensei Inti Escobar, que dirige el Dojo Bushido Kai de Cochabamba, consiste en  Recreación de un torneo de Karate no viable, por que
3.- Respuesta sencilla es sí, pero es mucho más complicado que eso, después de investigar un poco, se plantea el uso de controles con sensor de movimiento, sin embargo, la disponibilidad y los precios dificulta su alcance, ademas que pueden no soportar los movimientos bruscos.
4si .- Debido a esto se plantea el uso de cámaras o Kinect, sin embargo, kinect es un equipo caro que no tuvo grandes éxitos, por lo que es normal que nadie vaya a usarlo, así que se plantea el uso de cámaras normales




A causa del Coronavirus, hasta el día 1 de Octubre se han contagiado ya 34 millones de personas y hubo 1 millón de muertes, el cual esta causando pánico y desorden público. La Organización Mundial de la Salud declaro al Coronavirus como una emergencia de salud internacional. En Bolivia hay ya 135,311 enfermos, de los cuales 7.965 han muerto y hay todavía 31.817 casos activos, además de 95.529 dados de alta, con un 5,89\% de muerte. Eventualmente se tomaron medidas, las cuales son fuertemente criticadas independientemente de la calidad de resultados, tiempo durante el cual se ha estado viendo sofocadas las posibilidades de ir afuera libremente y comunicarse normalmente. \\

Este problema provoco que las ciudades del mundo impongan leyes de confinamiento, implementando advertencias, multas y cárcel para quienes no lo cumplan, así mismo, se cancelaron los días festivos nacionales, se cerraron escuelas y pospusieron clases, incluso los negocios que no eran considerados necesarios para la primera necesidad tuvieron que cerrar temporalmente durante largo tiempo hasta que las personas adoptaron  costumbres de higiene personal. El deporte ha sufrido especialmente debido a la influencia del virus como nunca antes ha acontecido, por ello las personas, especialmente los deportistas precisan de mantener sus rutinas diarias de ejercicio físico y las empresas deben planear nuevos modelos de negocio en orden de ajustarse a los cambios.\\

Se estima que en general la falta de actividad física prolongada puede inducir a gastar un tiempo exagerado en sentarse o echarse realizando distintas actividades (como usar el celular ver televisión, jugar videojuegos); así como evitar todo contacto con la actividad física, ocasionara una reducción de energía y  un aumento en las probabilidades de contraer enfermedades crónicas o complicaciones de salud. Se requiere mantener el ritmo de la actividad física para el sistema inmune funcionando a una capacidad estándar/optima.\\

El ejercicio en casa puede realizarse de demasiadas formas y muy variadas, basta con buscar un ambiente que facilite aludir el virus y sea un espacio mínimamente grande para entrar echado cómodamente con los brazos y piernas extendidos, aunque no sean de las mejores practicas, es incluso suficiente un par de metros cuadrados para entrenar en distintos deportes o disciplinas. Se pueden exhibir múltiples ejemplos de ejercicio, por lo que es casi imposible tener una excusa, la población que no puede darse el lujo de ejercitarse, esta en una situación deplorable económicamente, carece del tiempo debido a un exceso insano de trabajo, problemas de salud, etc.
Se pueden realizar entrenamientos desde flexiones, sentadillas, abdominales, trotar en el mismo lugar y hacer juego de pies en el mismo lugar, así mismo, se puede practicar Yoga, Tai Ji Quan, Karate y diversas artes marciales que poseen técnicas y practicas tanto a ambientes grandes como a pequeños y puede ser practicado en cualquier momento, solo con la presencia de voluntad y un tiempo bien administrado. Existen innumerables vídeos y guías de ejercicio para realizar en Internet y televisión e incluso juegos o aplicaciones para poder ejercitarse lo necesario o ir más allá.\\

Lo anterior mencionado no significa que el deporte o el ejercicio deba ser obligatoriamente limitado o que se deben restringir el ejercicio si no se cumple con estas restricciones, sino que esta es una medida más en contra de la expansión del virus hasta que exista una vacuna o resistencia efectiva ante ella. Se recomienda hacer ejercicio durante al menos 30 minutos cada día o al menos 20 minutos de exigencia vigorosa. En cuanto a niños, ancianos o enfermos crónicos, consultar con un medico es recomendado. Mantenerse en casa es efectivo para evitar la expansión de esta enfermedad, pero mantener la actividad física es un punto importante.\cite{Chen} \\



La enseñanza de entrenadores a deportistas siempre ha sido a partir de ordenes y tutoriales cuya base teórica es el enfoque cognitivo. Esto significa que la experiencia personal y el entrenamiento previo o visualización de resultados de los entrenadores se verá reflejado en su metodología de enseñanza, señalizando principalmente ordenes de ejecución de movimiento, entrenamiento mental (cuyo objetivo es desarrollar la autoestima y pensamiento/visualización positiva) y la retroalimentación para corregir errores y perfeccionar la técnica.\\



Debido a las medidas constantemente tomadas por el gobierno regional y nacional, la nueva normalidad que nos permita interactuar fuera de casa impone ciertas limitaciones al contacto y




































