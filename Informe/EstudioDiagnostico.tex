\chapter{Estudio de Diagnóstico}


Esta propuesta nace del Sensei Inti Escobar, que dirige el Dojo Bushido Kai de Cochabamba, donde practico el arte marcial (Ricardo Fernández), es algo que me apasiona. Debido a la situación delicada del Corona Virus, los deportes han visto muchos problemas, entre ellos el Karate Deportivo necesita más opciones para evolucionar y adaptarse a estos cambios.
A partir de ahí, el Sensei Inti planteo la siguiente pregunta “¿Existe la posibilidad de que usando equipos tecnológicos podamos recrear un torneo de Kata de Karate?” y la respuesta sencilla es sí, pero es mucho más complicado que eso, después de investigar un poco, se plantea el uso de controles con sensor de movimiento, sin embargo, la disponibilidad y los precios dificulta su alcance. Debido a esto se plantea el uso de cámaras o Kinect.


1.- Debido a la situación delicada del Corona Virus, los deportes han visto muchos problemas, entre ellos el Karate Deportivo necesita más opciones para evolucionar y adaptarse a estos cambios.
2.- Esta propuesta nace del Sensei Inti Escobar, que dirige el Dojo Bushido Kai de Cochabamba, consiste en  Recreación de un torneo de Karate no viable, por que
3.- Respuesta sencilla es sí, pero es mucho más complicado que eso, después de investigar un poco, se plantea el uso de controles con sensor de movimiento, sin embargo, la disponibilidad y los precios dificulta su alcance, ademas que pueden no soportar los movimientos bruscos.
4si .- Debido a esto se plantea el uso de cámaras o Kinect, sin embargo, kinect es un equipo caro que no tuvo grandes éxitos, por lo que es normal que nadie vaya a usarlo, así que se plantea el uso de cámaras normales


Descripción estructurada detallada de la situación
problemática que origina la necesidad del proyecto
de grado.
10-20 paginas

La enseñanza de entrenadores a deportistas o siempre ha sido enseñado a partir de ordenes y tutoriales cuya base teórica es el enfoque cognitivo. Esto significa que la experiencia personal y el entrenamiento previo o visualización de resultados de los entrenadores se verá reflejado en su metodología de enseñanza, señalizando principalmente ordenes de ejecución de movimiento, entrenamiento mental (cuyo objetivo es desarrollar la autoestima y pensamiento/visualización positiva) y la retroalimentación para corregir errores y perfeccionar la técnica.




A causa del Coronavirus, hasta el día 30 de septiempre se han contagiado ya 34 millones de personas y hubo 1 millon de muertes, 



