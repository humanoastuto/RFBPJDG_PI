\chapter{Estudio de alternativas}

\section{Análisis de Alternativas actualmente disponibles en el mercado}

En la actualidad, en la extensión de la investigación, existen productos comerciales y proyectos privados que se asemejan a las características ofrecidas por el proyecto, son en su mayoría de índole privatizada especialmente en el campo de la medicina, donde solo se hace mención a sus resultados, sin embargo, ninguna resalta por su accesibilidad gratuita, distinguiendo el proyecto del factor común. Siendo las aplicaciones y programas desarrollados con un enfoque principal en entretenimiento, como es el caso de videojuegos y medicina, en analizadores médicos y programas de rehabilitación.\\

\subsection{Entretenimiento}

Cuando se referencia a un sistema de detección de pose del cuerpo, se nombran varios videojuegos de baile, siendo las primeras y más importantes sagas, las desarrolladas por Ubisoft como Just Dance o Dance Experience, enfocadas en realizar coreografías pre-diseñadas para canciones populares en la época del lanzamiento de sus entregas; a pesar de que en Just Dance han existido niveles con temática de artes marciales, estas no contaban con la intención de ser un entrenamiento educativo, sino un baile con el propósito de entretener utilizando movimientos basados del arte marcial.  \\

Además de las mencionadas, existe otro género en los videojuegos enfocado al entrenamiento físico, como Shape Up de Ubisoft y Wii Fit de Nintendo, estas enfocadas más a un entrenamiento físico casual como es el caso de Shape Up, donde se incentiva al jugador a realizar ejercicios anaeróbicos de alta intensidad como son flexiones, abdominales o sentadillas y Wii Fit enfocado a rutinas de Yoga, equilibrio y aeróbicos; demostrando las diferentes aplicaciones y posibilidades de desarrollo y resultados. \\

\subsubsection{Just Dance}

Un éxito comercial e inspiración del proyecto es este sistema interactivo, que emplea al máximo el controlador Kinect y el Body Tracking con comandos de voz para ofrecer un uso casual y entretenido. Se espera que el equipamiento provisto para el proyecto, carente de una cámara de profundidad (elemento sobresaliente en el controlador Kinect), permita el desarrollo adecuado de las características de Body Tracking.

\subsection{Analizadores Médicos}

En esta rama, la información es más privatizada o proporciona un menor alcance al publico, no obstante, se ha encontrado que se utiliza la tecnología planeada para controlar la pose del usuario, ya sea en pruebas médicas para hallar anormalidades físicas en la posición del cuerpo al realizar diversas actividades y en entrenamiento físico para mantener una posición estable y evitar dañarse a uno mismo.

\section{Análisis de Alternativas En Herramientas Disponibles Para el Desarrollo}


\subsubsection{TensorFlow}

TensorFlow es una herramienta open source para el aprendizaje automático provista de soporte por Microsoft, la comunidad la ha empleado extensamente en proyectos y estudios de Body Tracking con el uso de PoseNet, derivado de TensorFlow, cuenta con soporte y una documentación clara, siendo además sus requisitos recomendados para su uso relativamente bajos.

\subsubsection{wrnch}

Una herramienta de calidad para el desarrollo de aplicaciones con Body Tracking, posee un gran potencial para el desarrollo del proyecto, contando con el esqueleto que se forma al seguir los movimientos de la persona, además cuenta con una opción Multi-Cam, capaz de seguir el movimiento de los dedos al mismo tiempo que el cuerpo completo casi en tiempo real. Como debilidad, la dependencia del Hardware y sus elevados requisitos para emplear al máximo esta herramienta con el equipamiento disponible, limitó sus posibilidades y uso en el proyecto.





Tensorflow

\href{https://www.tensorflow.org/lite/models/pose_estimation/overview}{Tensorflow}


Posenet parece ser la herramienta por defecto puesto que tiene mucho soporte por Google y la comunidad hizo muchos trabajos de body tracking con PoseNet de Tensorflow asi que documentacion hay de sobra, los incovenientes serian que es bastante pesado para la computadora y la instalacion no es nada sencilla.

wrnch
https://wrnch.ai/
Son uno de los mas avanzados en este ambito, hay varias demos en las que se puede ver un gran potencial, tienen el tipico esqueleto que sigue tus movimientos con una camara pero tambien tienen un proyecto con Multi-Cam que es capaz de seguir el movimiento de los dedos al mismo tiempo que el cuerpo completo casi en tiempo real, claro que esto ultimo depende mucho del hardware, aun asi es una gran posible herramienta para lograr nuestro objetivo.

Openpose
https://github.com/CMU-Perceptual-Computing-Lab/openpose
Similar a Posenet es gratis y tiene bastantes implementaciones, sigue en constante actualizacion, tiene APIs para python y C++ aunque es un poco complicado instalarlas, un plugin para Unity algo desactualizado, no solo reconoce la postura sino que cuenta con opciones para reconocer el rostro y manos.

DeepMotion
https://deepmotion.com/3d-body-tracking
Lo especial de este producto es que no necesita muchos requerimientos de hardware y parece ser bastante preciso, lamentablemente no es muy facil emplear esta tecnologia puesto que no es gratis, se intento comunicar con ellos pero lamentablemente la compañia sigue sin responer nuestra solicitud.

Unity kinect
Es una opcion con la que resulta más sencilla dearrollar la aplicacion porque no es necesario hacer un escaneo muy complicado del cuerpo por las herramientas que ya posee kinect, pero no es muy comun tener un kinect hoy en dia porque la misma Xbox (desarrolladora del kinect) dejo de darle apoyo al punto que sus ultimas consolas de videojuegos ya no incluyen el adaptador para este dispositivo olvidado.

Azure kinect
https://azure.microsoft.com/en-us/services/kinect-dk/
Es un muy sofisticado dispositivo con multiples sensores que detectan los bojetos con mayor presicion que un kinect normal, seria una alternativa excelente si no fuera por el precio del Azure Kinect DK que son unos 400dolares, y aunque consigamos uno para el desarrollo de la aplicación el cliente tendria que invertir mucho para jugar un solo juego.

Trajes con sensores de movimiento
Que mejor para detectar los movimientos que muchos sensores pegados a tu cuerpo, estos trajes con sensores son muy utilizados en el desarrollo de videojuegos pero en la etapa de desarrollo y no es muy comun ponerse a jugar con ellos, hay muchas alternativas en este campo como Xsens, Holosuit, Teslasuit, etc. Son más precisos que un programa que usa una camara o incluso un kinect, el incoveniente otra vez es el precio y que en muchos casos el traje tendria que ser hecho a medida ya que es poco probable que todos los clientes tengan las mismas tallas.

Crear software propio para la captura de momvimiento
Si lo hicieramos por cuenta propia entenderiamos como funciona a la perfeccion, lo que puede facilitar el manejo de posible herramienta aunque tomaria mas tiempo, no estaria tan refinado como las otras alternativas ya que tienen la experiencia que nos falta y no estariamos implementando lo aprendido en Ingenieria de Software.











\subsection{borrar esta parte}
Determinación de las especificaciones a cumplir
Identificación de alternativas comerciales de
sistemas, equipos o piezas
Definición de criterios de evaluación
Selección de las alternativas óptimas
5 hojas


Alternativas que podemos usar o se refiere a alternativas que existen aparte de nuestra idea

Buenos días Inge, estaba empezando a redactar el estudio de alternativas y tenía una duda respecto a la interpretación del estudio, si es en realidad:
Alternativas que existen en el mercado y por que deberían escoger nuestra solución y que son diferentes de los demás o las herramientas que podríamos haber utilizado y por que escogimos las que utilizamos.


osea debes hacer un estudio de que herramientas existen en el mercado y cuales son las mejores para tu implementación