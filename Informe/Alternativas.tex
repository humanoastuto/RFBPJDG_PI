\chapter{Estudio de alternativas}

\section{Análisis de Alternativas actualmente disponibles en el mercado}

En la actualidad, en la extensión de la investigación, existen productos comerciales y proyectos privados que se asemejan a las características ofrecidas por el proyecto, son en su mayoría de índole privatizada especialmente en el campo de la medicina, donde solo se hace mención a sus resultados, sin embargo, ninguna resalta por su accesibilidad gratuita, distinguiendo el proyecto del factor común. Siendo las aplicaciones y programas desarrollados con un enfoque principal en entretenimiento, como es el caso de videojuegos y medicina, en analizadores médicos y programas de rehabilitación.\\

\subsection{Entretenimiento}

Cuando se referencia a un sistema de detección de pose del cuerpo, se nombran varios videojuegos de baile, siendo las primeras y más importantes sagas, las desarrolladas por Ubisoft como Just Dance o Dance Experience, enfocadas en realizar coreografías pre-diseñadas para canciones populares en la época del lanzamiento de sus entregas; a pesar de que en Just Dance han existido niveles con temática de artes marciales, estas no contaban con la intención de ser un entrenamiento educativo, sino un baile con el propósito de entretener utilizando movimientos basados del arte marcial.  \\

Además de las mencionadas, existe otro género en los videojuegos enfocado al entrenamiento físico, como Shape Up de Ubisoft y Wii Fit de Nintendo, estas enfocadas más a un entrenamiento físico casual como es el caso de Shape Up, donde se incentiva al jugador a realizar ejercicios anaeróbicos de alta intensidad como son flexiones, abdominales o sentadillas y Wii Fit enfocado a rutinas de Yoga, equilibrio y aeróbicos; demostrando las diferentes aplicaciones y posibilidades de desarrollo y resultados. \\

\subsubsection{Just Dance}

Un éxito comercial e inspiración del proyecto es este sistema interactivo, que emplea al máximo el controlador Kinect y el Body Tracking con comandos de voz para ofrecer un uso casual y entretenido. Se espera que el equipamiento provisto para el proyecto, carente de una cámara de profundidad (elemento sobresaliente en el controlador Kinect), permita el desarrollo adecuado de las características de Body Tracking.

\subsection{Analizadores Médicos}

En esta rama, la información es más privatizada o proporciona un menor alcance al publico, no obstante, se ha encontrado que se utiliza la tecnología planeada para controlar la pose del usuario, ya sea en pruebas médicas para hallar anormalidades físicas en la posición del cuerpo al realizar diversas actividades y en entrenamiento físico para mantener una posición estable y evitar dañarse a uno mismo.


\subsubsection{TensorFlow}

TensorFlow es una herramienta open source para el aprendizaje automático provista de soporte por Microsoft, la comunidad la ha empleado extensamente en proyectos y estudios de Body Tracking con el uso de PoseNet, derivado de TensorFlow, cuenta con soporte y una documentación clara, siendo además sus requisitos recomendados para su uso relativamente bajos.

\subsubsection{wrnch}

Una herramienta de calidad para el desarrollo de aplicaciones con Body Tracking, posee un gran potencial para el desarrollo del proyecto, contando con el esqueleto que se forma al seguir los movimientos de la persona, además cuenta con una opción Multi-Cam, capaz de seguir el movimiento de los dedos al mismo tiempo que el cuerpo completo casi en tiempo real. Como debilidad, la dependencia del Hardware y sus elevados requisitos para emplear al máximo esta herramienta con el equipamiento disponible, limitó sus posibilidades y uso en el proyecto.

















\subsection{borrar esta parte}
Determinación de las especificaciones a cumplir
Identificación de alternativas comerciales de
sistemas, equipos o piezas
Definición de criterios de evaluación
Selección de las alternativas óptimas
5 hojas


Alternativas que podemos usar o se refiere a alternativas que existen aparte de nuestra idea

Buenos días Inge, estaba empezando a redactar el estudio de alternativas y tenía una duda respecto a la interpretación del estudio, si es en realidad:
Alternativas que existen en el mercado y por que deberían escoger nuestra solución y que son diferentes de los demás o las herramientas que podríamos haber utilizado y por que escogimos las que utilizamos.


osea debes hacer un estudio de que herramientas existen en el mercado y cuales son las mejores para tu implementación