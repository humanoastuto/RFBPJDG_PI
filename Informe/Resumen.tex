\chapter*{\begin{center} \begin{normalsize}Resumen Ejecutivo\end{normalsize}\end{center}}
	
\begin{quotation}
	\noindent El constante desarrollo de la tecnología crea oportunidades de expandir posibles resoluciones a los mismos problemas con métodos distintos, que con el tiempo mejoran o requieren menos capacidades. 
	
	Los sistemas interactivos son una gran alternativa para varias tareas que los usuarios comunes puedan resolver con su uso, en este proyecto se presenta un prototipo de un sistema interactivo con Body Tracking para que el usuario sea capaz de generar coreografías o pasos a seguir que otro usuario sea capaz de emplear para imitar las acciones que se realicen. Se la considera como una alternativa a sistemas interactivos como Just Dance o Shape up, los cuales son dirigidos al sector de entretenimiento, en baile y entrenamiento. Como característica adicional, esta alternativa es libre de la necesidad de una cámara de profundidad de campo, como un Kinect o un PlayStation Camera, los cuales son comúnmente usado para estas actividades. 
	El prototipo esta basado en OpenPose, una API Open Source disponible para el uso con lenguaje C Sharp y Python, así mismo se emplearan conceptos de Point Cloud para la comparación de las poses y Unity, tanto para la creación de una interfaz gráfica con el principio KISS y la unión base del proyecto.
\end{quotation}	
\clearpage

\chapter*{\begin{center} \begin{normalsize}Abstract\end{normalsize}	\end{center}}

\begin{quotation}
	\noindent Constant development of technology creates opportunities to expand possible solutions to the same problems with different methods, which over time improve or require fewer capabilities.
	
	Interactive systems are a great alternative for various tasks that regular users can solve with their use, in this project a prototype of an interactive system with Body Tracking is presented so that the user is able to generate choreographies or steps to follow so that other user is able to use it to imitate the actions that are carried out. It is considered as an alternative to interactive systems such as Just Dance or Shape up, which are aimed at the entertainment, dance and training sectors. As an additional feature, this alternative is free of the need for a depth of field camera, such as a Kinect or a PlayStation Camera, which are commonly used for these activities.
	The prototype is based on OpenPose, an Open Source API available for use with C Sharp and Python language, as well as Point Cloud concepts to compare the poses and Unity, both for the creation of a graphical interface with the principle KISS and the base union of the project.
\end{quotation}
\clearpage
