\chapter{Resumen}

Resumen breve y conciso que informa el contenido
del trabajo de grado. Se recomienda que esté
constituido por la presentación de los antecedentes
y su importancia, la formulación de los objetivos, la
descripción del método o procedimiento y la
presentación de los resultados obtenidos.
Debe redactarse en los idiomas español e inglés
1-2 paginas







Papel
Bond, tamaño Carta (215.9 X 279.4 mm).
Impresión en ambas caras de la hoja (Opcional)
Procesador de Texto MS Word o similar
Redacción
El lenguaje de los documentos académicos es opuesto al lenguaje literario. Es
directo, claro, preciso y conciso. La Redacción es impersonal, en tercera persona
singular. Se debe evitar la utilización de la primera persona, poner exclamaciones y
hacer preguntas en primera persona. También se debe respetar la coherencia en los
tiempos verbales. El texto debe ser redactado, en lo posible, en tiempo pretérito.
Las unidades físicas y las abreviaciones empleadas deben ser las oficiales. Para
usuarios del Sistema Métrico internacional, se recomienda consultar:
http://physics.nist.gov/Pubs/SP330/sp330.pdf [08.07.2011]
Guía recomendada de redacción:
http://edicionesdigitales.info/Manual/Manual/ingles.html [08.07.2011]




CONTENIDO
Lista de títulos empleados en el documento, con indicación de la página en que aparecen,
generada automáticamente por MS Word, una vez que los estilos de título han sido definidos
en todos los niveles necesarios y se han marcado correctamente en el documento
(Referencias, Tabla de Contenido, Insertar Tabla de Contenido, Seleccionar Formato y
Niveles a mostrar y Aceptar)
MS Word no actualiza automáticamente la tabla de CONTENIDO. Normalmente,
actualizarla es la última operación que se hace antes de concluir el documento (Con el cursor
sobre la tabla, presionar el botón derecho del mouse, seleccionar Actualizar campos,
seleccionar Actualizar toda la Tabla y Aceptar)
Lista de tablas
Lista de Tablas del documento, incluyendo las de los Anexos, con indicación del número
correlativo, el título y la página en que aparecen.
Generada automáticamente por MS Word con los comandos de Insertar Título e Insertar
Tabla de ilustraciones, del menú Referencias
Numeración correlativa a lo largo de todo el documento, simple (Nº 14), o por capítulos
principales (Nº 3.12)
Lista de figuras
Lista de Figuras del documento, incluyendo las de los anexos, con indicación del número
correlativo, el título y la página en que aparecen.
Generada automáticamente por MS Word con los comandos de Insertar Título e Insertar
Tabla de ilustraciones, del menú Referencias
Numeración correlativa a lo largo de todo el documento, simple o por capítulos principales
Lista de
abreviaciones
(opcional)
Lista de abreviaciones y acrónimos empleados. Sólo cuando su número es importante.
Glosario (opcional) Descripción de los conceptos más importantes empleados, cuando es imprescindible que los
lectores empleen el mismo lenguaje que el autor
Dedicatoria
(opcional)
Nota mediante la cual se dedica el trabajo a una o varias personas o instituciones (opcional).
Debe escribirse en el extremo inferior derecho de una página nueva y no exceder las cinco
líneas. Ejemplos de la mejor opción: A mis padres, A María
Página de
agradecimientos
(opcional)
Como capítulo principal, debajo del título centrado: AGRADECIMIENTO. Sólo para
mentores, colegas, financiadores y personas o instituciones que han apoyado el trabajo más
allá de lo profesionalmente esperado


Texto: Capítulos
Los capítulos deben iniciar una nueva página
El texto de los títulos no admite mayúsculas en las palabras, salvo que sea la que
inicia la frase o que se trate de un nombre propio, siguiendo las últimas reglas de la
Real academia de la lengua española.




Títulos
Se establece el siguiente formato:
Nivel de Título Tamaño (puntos) Mayúsculas o Minúsculas Negrilla Cursiva Subrayado
Títulos de
carátula
18 MAYÚSCULAS Sí No No

Título del
capítulo
(Título 1)
16 MAYÚSCULAS Sí No No

Título
principal
dentro del
capítulo
(Título 2)
14 Minúsculas tipo
Título
Sí No No

Primer título
secundario
(Título 3)
12 Minúsculas tipo
Título
Sí No No

Segundo
título
secundario
(Título 4)
12 Minúsculas tipo
Título
Sí Sí No

Los títulos de Capítulo tienen numeración romana. Los otros niveles de título tienen
numeración arábiga. Los diferentes niveles se separan con puntos. El último nivel
no tiene punto detrás de la cifra.
Además, mantener el formato general también para cada nivel de título:
Interlineado de 1.5, 12 puntos anteriores, 12 puntos posteriores, 0 cm de sangría
izquierda y 0 cm de sangría derecha.
Se recomienda no emplear más de 3 niveles de títulos, dentro de cada capítulo.



Tablas
Las tablas son estructuras de filas y columnas que deben estar necesariamente
referenciadas en el texto. Están precedidas por un encabezado, que muestra el
número correlativo de la tabla y su título. La primera fila de la tabla contiene los
títulos de cada una de las columnas. Las filas siguientes contienen la información.
El encabezado de las tablas emplea Times New Roman como fuente, tamaño 10,
está en negrilla y como atributos de párrafo el interlineado sencillo, 6 puntos
anteriores y 6 puntos posteriores.
Si la Tabla ha sido construida en base a información secundaria obtenida de
terceros, debe registrarse una cita bibliográfica debajo: Fuente: [Cita bibliográfica],
con el mismo formato establecido para el encabezado: Times New Roman, tamaño
10, Interlineado sencillo, 6 puntos anteriores y 6 puntos posteriores.
Es recomendable aumentar la legibilidad de las tablas con filas sombreadas
intercaladamente. Para esto se debe usar el tono de gris más tenue posible, como el
de la presente tabla.
Para asegurar la generación de la Lista de Tablas, el título de cada una debe ser
registrado: En el menú Referencias, Insertar título, Seleccionar rótulo de tabla y
Aceptar. MS Word asigna automáticamente el número que corresponde y lo
actualiza en caso de cambios. Verificar que el Estilo Epígrafe, que almacena el
formato del encabezado de las tablas, cumpla lo arriba establecido.
La alineación del texto en las tablas es a la izquierda. La de los números, a la
derecha. La alineación ajustada a ambos márgenes no es conveniente.
No utilizar viñetas ni numeración en tablas de dos columnas o más.
Las tablas tomadas de terceros deben ser reconstruidas, para evitar el aspecto de
objetos cortados y pegados.
Figuras
Las figuras contienen información gráfica, además de alfanumérica.
También deben estar referenciadas en el texto y ser dotadas de un número
correlativo y un título, al pie de la Figura.
Si la figura ha sido construida con información de terceros, debe contar con la
correspondiente cita bibliográfica, en el mismo formato que el especificado para las
Tablas.
El pie de la figura y, si es necesario, la cita bibliográfica, se construyen en el mismo
formato que el encabezado y la cita de las Tablas.
Para asegurar la generación de la Lista de Figuras, el título de cada una debe ser
registrado: En el menú Referencias, Insertar Título, Seleccionar Rótulo Ilustración
y Aceptar. MS Word asigna automáticamente el número que corresponde y lo
actualiza en caso de cambios.



Citas y referencias
bibliográficas
Cada idea tomada de terceros en el documento, debe estar rigurosamente
referenciada empleando una herramienta de gestión bibliográfica.
Gestores bibliográficos avanzados son Zotero, Mendeley u otros.
Por ser el más común, puede emplearse la herramienta “Citas” de MS Word de la
siguiente manera:
En computadoras PC: En el menú Referencias, seleccionar primero ISO 690-
Primer elemento y Fecha o ISO 690 – Referencia numérica. Luego, con el cursor en
el punto donde debe introducirse la cita, seleccionar Insertar cita, Agregar nueva
fuente, Seleccionar Tipo de fuente Bibliográfica, llenar la información exigida por
la norma ISO 690 y Aceptar
El botón Administrar fuentes, del menú Referencias, Citas y Bibliografía, permite
usar referencias ya disponibles o realizar modificaciones citada,
En computadoras Apple: Ícono de Herramientas (si no lo tiene, hacer clic en Ver
> Barras de herramientas > Estándar) > Citas > Seleccionar Estilo de cita: APA >
Ícono de + si es primera vez que se emplea la cita > Elegir tipo de fuente
bibliográfica > Llenar como mínimo los datos marcados con asterisco > Aceptar.
Para usar referencias ya registradas: Ícono de Herramientas > Citas > Elegir la
fuente bibliográfica > Aceptar.
Cita directa o textual
Las citas textuales en párrafos se escriben dentro del texto entre comillas y en
cursiva, con márgenes izquierdo y derecho 1 cm mayores a los adoptados por el
documento, 3.5 cm (izquierda) y 2.5 cm (derecha) en este caso, de tal manera que el
texto citado no llega a ocupar el espacio que ocupa el texto normal
Notas a pie de página
Notas, explicaciones o traducciones del autor, para precisar el texto
Fuente: Times New Roman, 9 puntos
Párrafo: Interlineado sencillo, 0 puntos anteriores y 0 puntos posteriores




