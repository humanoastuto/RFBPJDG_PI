\chapter{Introducción}

\section{Antecedentes}

Durante 10 años, el desarrollo de sistemas interactivos con mira en el Body Tracking se encuentra en el mercado global, siendo una herramienta muy popular para la actividad física en casa, la rehabilitación de pacientes postrados en cama durante un largo tiempo y por supuesto, para el entretenimiento de los consumidores.

A nivel Bolivia, el desarrollo de sistemas interactivos es comúnmente dejado de lado, debido a la creencia banal de ser gravoso y poco gratificante, también de la necesidad de los habitantes de desarrollar diversos sectores más beneficiarios para la población y economía. Sin embargo, el desarrollo de sistemas interactivos con Body Tracking pueden ser aplicados a muchas áreas, potencialmente en la medicina, la educación y especialmente en el entretenimiento.

El constante desarrollo de la inteligencia artificial y Machine Learning revoluciono la forma de desarrollar Software, la liberación de información e inclusión de su educación, kits de desarrollo de Software, alternativas Open Source y librerías que permiten su uso en distintas áreas pueden ser ampliamente usadas para cumplir con las expectativas de los usuarios, buscando alternativas más accesibles. 

\section{Descripción del Problema}

La brecha tecnológica y económica entre los países del mundo, la diferencia entre las culturas acostumbradas a la tecnología y los países en vías de desarrollo, sumado a la actitud de la población de países en vías de desarrollo sobre la tecnología, ve como dispendioso la compra de un Kinect o una cámara con esas características (profundidad de campo y sensores RGB-D) ya que en su mayoría, los usuario no la necesitan ni la emplean para múltiples tareas, solo para el entretenimiento, por lo que termina acumulando polvo.

La visión inicial del proyecto fue obstruida por este motivo, ya que se quería desarrollar un sistema interactivo empleando Kinect, sin embargo, esto no fue posible, por tanto, la perspectiva se debe centrar en la falta de herramientas comunes para el desarrollo de sistemas interactivos con Body Tracking, función que volvió comercial a Kinect y explorar la posibilidad de no necesitar una cámara especial que solo se usará para una actividad y posiblemente sea desechado al dejar de usarlo.

Existen varios sistemas interactivos con Body Tracking, como Shape Up o Just Dance, que van dirigidos a públicos distintos, pero si bien existe la posibilidad de un sistema interactivo que los una, no es conocido o un éxito en ventas, por tanto se requiere de una alternativa que solucione la necesidad de contar con varios programas diferentes para varias actividades distintas.

\section{Justificación}

El presente proyecto busca analizar las distintas herramientas que existen y puedan proporcionar una alternativa asequible y funcional que cumplan con los requerimientos que se esperan. Adicionalmente, busca dar un aporte a las funciones dadas por distintas alternativas para la atracción de los usuarios a esta alternativa.

En Bolivia, el uso de la inteligencia artificial puede ser la clave para mitigar el costo del desarrollo de sistemas interactivos con Body Tracking y conlleve a extender su uso y popularidad. Un primer paso puede ser el cambio de perspectivas de la población que puede permitirse estos dispositivos a través de un sistema interactivo con Body Tracking dirigido al entretenimiento, proporcionando una nueva alternativa con una visión diferente a la del común como es Just Dance o Shape It.

Existe una necesidad de promover y crear nuevas opciones para diferentes tareas y el Body Tracking tiene múltiples posibilidades, dándolas a conocer, estudiando y produciendo resultados, se generará en el futuro atención a este prometedor campo.

\section{Delimitación}

El presente proyecto de práctica interna limitará su presentación de diseño y desarrollo a una versión beta con únicamente el producto mínimo viable, que consta de unicamente el desarrollo de las partes vitales que debe constituir el sistema interactivo para ser considerado como un desarrollo exitoso. 

Debido a la naturaleza del proyecto y los recursos disponibles, no se garantiza que el producto desarrollado sea asequible para toda la población, mas bien busca atraer a las personas interesadas a ver el potencial que presenta este campo del desarrollo y decidan invertir en ello, así como busca atraer a usuarios que al emplearlo cambien su perspectiva respecto al tema.

Debido a la naturaleza de las herramientas empleadas y su constante mejora y continuo crecimiento que prosigue hoy en día, es probable que existan errores con los que no se pueda lidiar dentro de la versión beta presentada.


