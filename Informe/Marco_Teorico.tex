\chapter{Marco Teórico}

\section{Antecedentes}

	El karate deportivo es una rama que requiere de practica y corrección para una mejora continua, existen dos formas de competir, en kata y kumite.
Kumite es combate marcado con puntos y reglas para no lastimar y la Kata es la representación de técnicas ejecutadas en orden para demostrar tu habilidad y perfección de las técnicas aprendidas a lo largo de los años. 
Este arte es tradicional, por lo que el prejuicio de aprender desde internet no se considera lo suficientemente adecuado para alcanzar la perfección. Sin embargo, existen panoramas y sucesos que pueden forzar esta situación, en este caso el coronavirus, al ser forzados a no poder ir a la clase de Karate y mucho menos salir de casa, uno tiene que buscar alternativas.
Este problema de aislamiento nos lleva a la necesidad de sobrellevar la crisis con tecnología, llevando el arte tradicional a virtual, donde actualmente se realizan las clases y una lluvia de ideas para extender estas prácticas. Una de estas ideas es desarrollar un juego multijugador para llevar a cabo un torneo entre los participantes, sin embargo, esto es poco práctico, siendo así, se plantea una alternativa.



Descripción estructurada de las bases teóricas,
leyes, normas, estándares, alternativas de solución,
mejores prácticas, supuestos y otras reglas,
adoptados para la realización del proyecto

5-10 paginas