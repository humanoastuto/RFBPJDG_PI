\chapter{Plan de Actividades}

El desarrollo del proyecto inicia oficialmente el día 17 de Septiembre, con la definición del perfil del proyecto por parte del grupo de estudiantes.

El desarrollo fue interrumpido debido a la concentración del equipo en otras actividades, que exigió el 100\% del tiempo libre y de estudio y progreso en el proyecto, siendo más específicos la materia de Aplicación con Redes, estimando entre 4 a 8 horas diarias (incluyendo fines de semana) de exigencia entre estudio y practicas para mantener el ritmo solo a esa materia. Debido a ello, desde la fecha 18 de septiembre del 2020, a la fecha 4 de noviembre del 2020, el progreso fue mínimo, reduciéndose a la búsqueda de API's y herramientas, así como la redacción base del proyecto y la oficialización interna del producto deseado.

La metodología SCRUM fue inicializada oficialmente y de manera constante el día 10 de Noviembre del 2020, planteando un rango de horas de trabajo de 2 a 5 horas diarias mínimo (incluyendo fines de semana), con la finalización del Sprint con la conclusión de las tareas designadas al Sprint, en caso de resultar imposible terminar a tiempo una tarea, se la arrastrará a la siguiente Sprint, forzando una finalización del Sprint hasta donde se encuentra el proyecto en ese momento.


\restoregeometry
\newgeometry{left=1.3cm,bottom=1.3cm}
\begin{landscape}
\setstretch{1}
	\subsection{Cronograma}
	\begin{ganttchart}[
		hgrid=true,
		vgrid={*1{red, dashed}, *1{green, dashed}, *1{blue, dashed}},
		bar/.append style={fill=red!50},
		x unit=2.5mm,
		time slot format=isodate
		]{2020-09-16}{2020-12-20}
		\gantttitlecalendar{month=name, , week=1, weekday} \\
		\ganttbar{T1}{2020-09-16}{2020-09-26} \\
		\ganttbar{T2}{2020-09-17}{2020-11-10} \\
		\ganttbar{T3}{2020-10-01}{2020-12-20} \\
		\ganttbar{T4}{2020-11-06}{2020-12-17} \\
		\ganttbar{T5}{2020-12-20}{2020-12-21} \\
		\ganttbar{T6}{2020-12-18}{2020-12-19} \\
		\ganttbar{T7}{2020-12-20}{2020-12-20} 		
	\end{ganttchart}

Cada TX representa un título
\begin{itemize}
	\item T1 \emph{Redacción del Perfil}.
	\item T2 \emph{Investigación}.
	\item T3 \emph{Redacción del documento}.
	\item T4 \emph{Implementación de la aplicación}.
	\item T5 \emph{Revisión final del documento}.
	\item T6 \emph{Revisión final de la aplicación}.
	\item T7 \emph{Redacción de la presentación}.
\end{itemize}
\setstretch{1.5}
\end{landscape}

\restoregeometry
