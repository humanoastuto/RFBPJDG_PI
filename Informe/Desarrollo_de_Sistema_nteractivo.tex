\chapter{Desarrollo e Implementación del Sistema Interactivo}

\section{Funcionamiento Base}



\subsection{Configuración del Entorno}

Para el desarrollo de la aplicación se empleara Unity y Visual Studio, ya que Visual Studio es una SDK que permite programar en C sharp, que tiene una biblioteca disponible que lo vincula a Unity para su desarrollo.

\subsection{Creación del Menú}

\subsubsection{Botones}

\subsubsection{Opciones}

\subsection{Visualizar Movimientos del Usuario y Poses a Imitar}

\subsection{Comparación entre dos Poses}

\subsubsection{Normalización de Puntos Clave/Point Set Registration}

\subsubsection{Exhibir la Precisión de la Comparación al Usuario}

\subsection{Control de Tiempo del Mapa}

\subsubsection{Limite de Tiempo para realizar la posición}

Se la califica y se debe proceder a la siguiente pose cuando el tiempo termine y así se mueva al nuevo tiempo limite de la nueva posición

\subsubsection{Tiempo de Duración del Mapa}



\section{Conversión de un Vídeo a un Mapa}

\subsubsection{Conversión de formato JSON a archivo .txt}

\subsubsection{Eliminación de Archivos .txt}

\subsubsection{Adición del límite de tiempo a las poses}



