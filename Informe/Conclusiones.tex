\chapter{Conclusiones}


Durante el trayecto del proyecto se realizaron diferentes actividades y seguimientos de las mismas, además de la aplicación de herramientas externas apropiadamente empleadas y una metodología ágil en su mayoría, si bien se tuvo dificultades de gran proporción con la estimación de tiempo, a fin de cuentas, no fue un factor que determine el fracaso del proyecto, es más, se afirma que los objetivos fueron cumplidos de acuerdo al producto mínimo viable establecido en delimitación.

En función a los objetivos específicos, a lo largo del desarrollo e implementación del prototipo diseñado, se llega a la conclusión siguiente.

Respecto al primer objetivo específico, utilizar Software existente para el seguimiento corporal utilizando una cámara, en el estudio de alternativas se observan varias herramientas de Software que posibilitaban el desarrollo del proyecto, al emplear OpenPose, la mejor alternativa, que sirve para el seguimiento corporal y se la empleo utilizando cámaras web, siendo las pruebas con el prototipo realizadas empleando diferentes modelos de cámara Web, se determina que el objetivo específico es cumplido satisfactoriamente.
\\

Respecto al segundo objetivo específico, implementar una función para registrar mapas de movimiento propios del usuario,


Respecto al tercer objetivo específico,proveer una alternativa factible al mercado de sistemas interactivos con Body Tracking tales como Just Dance.


Respecto al objetivo general.


Desarrollo de un sistema interactivo con Body Tracking, empleando una cámara común para entrenamiento, seguimiento de posiciones, técnicas de arte marcial y baile.

Presentación de las principales conclusiones
Relación de los objetivos propuestos y los objetivos
alcanzados