\chapter{Conclusiones}


Durante el trayecto del proyecto se realizaron diferentes actividades y seguimientos de las mismas, además de la aplicación de herramientas externas apropiadamente empleadas y una metodología ágil en su mayoría, si bien se tuvo dificultades de gran proporción con la estimación de tiempo, a fin de cuentas, no fue un factor que determine el fracaso del proyecto, es más, se afirma que los objetivos fueron cumplidos de acuerdo al producto mínimo viable establecido en delimitación.

En función a los objetivos específicos, a lo largo del desarrollo e implementación del prototipo diseñado, se llega a la siguiente conclusión.
\\

Respecto al primer objetivo específico, utilizar Software existente para el seguimiento corporal utilizando una cámara, en el estudio de alternativas se observan varias herramientas de Software que posibilitaban el desarrollo del proyecto, al emplear OpenPose, la mejor alternativa, que sirve para el seguimiento corporal y se la empleo utilizando cámaras web, siendo las pruebas con el prototipo realizadas empleando diferentes modelos de cámara Web, se determina que el objetivo específico es cumplido satisfactoriamente.

Respecto al segundo objetivo específico, implementar una función para registrar mapas de movimiento propios del usuario, fue el objetivo específico más conflictivo, pero a pesar de todos los problemas, con la supervisión de los roles distintivos oportunos, se redirigió adecuadamente el proyecto a su finalización, proyectando los requisitos del producto mínimo viable.

Respecto al tercer objetivo específico, proveer una alternativa factible al mercado de sistemas interactivos con Body Tracking tales como Just Dance, se determina a través de la precisión del juego, la dificultad de crear mapas y factores menores subjetivos por parte de los usuarios, como la creación de mapas y la dificultad de reconocer adecuadamente las poses del usuario, que si es una alternativa factible, sin embargo, no se aproxima al nivel de calidad que ofrece un sistema interactivo como Just Dance. 

Respecto al objetivo general, que especifica el desarrollo de un sistema interactivo con Body Tracking con una cámara común para múltiples propósitos, se considera exitoso. Uno de los mayores desafíos fue la falta de herramientas, característica principal del proyecto, a pesar de la simplitud de la UI, las funcionalidades principales del producto si cumplen con los requisitos críticos del usuario.
Por otro lado, el desarrollo de nuevas alternativas, la constante mejora de versión de la herramienta OpenPose aplicada, así como las posibilidades de mayor libertad legal de emplear distintas herramientas, podrían terminar por incrementar la calidad del producto final y su firmeza como una alternativa frente a otros sistemas interactivos.

Durante la elaboración y desarrollo del proyecto, existieron tres principales problemas que acontecieron y demostraron lo caótico que un proyecto puede llegar a ser, los problemas provinieron de el tiempo disponible durante un mes y medio, la estimación de horas requeridas para cada tarea y el conflicto de ideas en la elaboración de tareas.

Un problema que tuvo el inicio del proyecto y su fase temprana fue el cese repentino de las actividades, de tiempo disponible, debido a la inoportuna aparición de una actividad curricular dentro de la universidad, más específicamente la materia de Aplicación de Redes, los estudiantes carecieron del tiempo necesario para realizar el proyecto, ya que la materia exigió toda la atención que se tenía disponible, entre 4 a 8 horas.

Uno de los principales problemas es la infraestimación de tiempo al usar la herramienta de Estimación de Poker que derivo en que la mayoría de las tareas lleguen a triplicar su tiempo estimado, si bien, el  total de 500 horas de trabajos fue calculado, no se esperaba que realmente fuera a ser necesario como un tiempo cercano, este problema fue producido por el desconocimiento de las herramientas necesarias para desarrollarlo, en muchos aspectos, esta fue una experiencia totalmente nueva.

Un problema fue la mala asignación de tareas, producto de la negligencia en el seguimiento necesario de la metodología, siendo más especifico, miembros del SCRUM Team encontraron conflicto de ideas en el segundo objetivo específico, ya que durante el desarrollo de las tareas relacionadas a registrar mapas del usuario, el choque entre las ideas de implementación ya desarrolladas, el tiempo invertido en ello y la imposibilidad de enlazar sus resultados, retraso el proyecto y por poco colapsa debido al tiempo, una vez fue definido más claramente las labores a realizar por el SCRUM Master, el SCRUM Team se puso de acuerdo y el desarrollo volvió a encaminarse.

